	%  A simple AAU report template.
%  2011-12-04 v. 0.1.1
%  Copyright 2010-2011 by Jesper Kjær Nielsen <jkn@es.aau.dk>
%
%  This is free software: you can redistribute it and/or modify
%  it under the terms of the GNU General Public License as published by
%  the Free Software Foundation, either version 3 of the License, or
%  (at your option) any later version.
%
%  This is distributed in the hope that it will be useful,
%  but WITHOUT ANY WARRANTY; without even the implied warranty of
%  MERCHANTABILITY or FITNESS FOR A PARTICULAR PURPOSE.  See the
%  GNU General Public License for more details.
%
%  You can find the GNU General Public License at <http://www.gnu.org/licenses/>.
%
\documentclass[11pt,twoside,a4paper,openright,fleqn]{report}
\usepackage{etex} % Inserted to be able to use listings package. Should be placed right after the documentclass definition.
\usepackage{paralist}
 
%%%%%%%%%%%%%%%%%%%%%%%%%%%%%%%%%%%%%%%%%%%%%%%%
% Language, Encoding and Fonts
% http://en.wikibooks.org/wiki/LaTeX/Internationalization
%%%%%%%%%%%%%%%%%%%%%%%%%%%%%%%%%%%%%%%%%%%%%%%%

% Select encoding of your inputs. Depends on
% your operating system and its default input
% encoding. Typically, you should use
%   Linux  : utf8 (most modern Linux distributions)
%            latin1 
%   Windows: ansinew
%            latin1 (works in most cases)
%   Mac    : applemac
% Notice that you can manually change the input
% encoding of your files by selecting "save as"
% an select the desired input encoding. 
\usepackage[utf8]{inputenc}
% Make latex understand and use the typographic
% rules of the language used in the document.
\usepackage[english]{babel}
% Use the vector font Latin Modern which is going
% to be the default font in latex in the future.
\usepackage{lmodern}
% Choose the font encoding
\usepackage[T1]{fontenc}

% insert pdf package
\usepackage{pdfpages}
%%%%%%%%%%%%%%%%%%%%%%%%%%%%%%%%%%%%%%%%%%%%%%%%
% Graphics and Tables
% http://en.wikibooks.org/wiki/LaTeX/Importing_Graphics
% http://en.wikibooks.org/wiki/LaTeX/Tables
%%%%%%%%%%%%%%%%%%%%%%%%%%%%%%%%%%%%%%%%%%%%%%%%
% The standard graphics inclusion package
\usepackage{graphicx}
% Set up how figure and table captions are displayed
\usepackage{caption}
\captionsetup{%
  font=footnotesize,% set font size to footnotesize
  labelfont=bf % bold label (e.g., Figure 3.2) font
}
% Make the standard latex tables look so much better
\usepackage{array,booktabs}
\pdfoptionpdfminorversion=6				% Allows inclusion of pdf documents, of version 1.6 and higher
% Allows pdflatex to compile with .eps included, Egon 2013-05-22
\usepackage{epstopdf}


%%%%%%%%%%%%%%%%%%%%%%%%%%%%%%%%%%%%%%%%%%%%%%%%
% Page Layout
% http://en.wikibooks.org/wiki/LaTeX/Page_Layout
%%%%%%%%%%%%%%%%%%%%%%%%%%%%%%%%%%%%%%%%%%%%%%%%
% Enable arithmetics with length. Useful when
% typesetting the layout.
\usepackage{algorithm2e}

\usepackage{calc}

% Change margins, papersize, etc of the document
%\usepackage[a4paper]{geometry}
%\setlength{\oddsidemargin}{15.5pt} 
%\setlength{\evensidemargin}{15.5pt}
\usepackage[a4paper,inner=3.5cm,outer=2.5cm,top=4cm,bottom=4cm,pdftex]{geometry}

% Change the headers and footers
\usepackage{fancyhdr}
\pagestyle{fancy}
\fancyhf{} %delete everything
\fancyhead[RE]{\small\nouppercase\leftmark} %even page - chapter title
\fancyhead[LO]{\small\nouppercase\rightmark} %uneven page - section title
\fancyhead[LE,RO]{\thepage} %page number on all pages
%\fancyfoot[C]{--- Confidential ---}
\setlength{\headheight}{13.6pt} 
% Do not stretch the content of a page. Instead,
% insert white space at the bottom of the page
\raggedbottom
% Insert space between paragrapsh and dont indent first line:
% Used instead of setlength{\parskip}{10pt plus 1pt minus 1pt} and \setlength{\parindent}{0pt}
\usepackage{parskip}



% adjust space arround chapter, section, subsection, subsubsection
% http://mirrors.dotsrc.org/ctan/macros/latex/contrib/titlesec/titlesec.pdf
\usepackage{titlesec}
%\titlespacing*{\chapter}{0pt}{-50pt}{40pt}


%%%%%%%%								%%%%%%
%				COMENTAT PER TINDRE				%
%				ESPAI DESPRES DE SUBSECTION		%
%%%%%%%%								%%%%%%

%\titlespacing\subsection{0pt}{4pt plus 2pt minus 2pt}{-4pt plus 2pt minus 2pt}
%\titlespacing\subsubsection{0pt}{4pt plus 2pt minus 2pt}{-6pt plus 2pt minus 2pt}

%%%%%%%%								%%%%%%
%						END COMMENT				%
%%%%%%%%								%%%%%%

%Define new chapter style - By Mads - 28 May 2013
\titleformat{\chapter}
  {\normalfont\Huge\bfseries}{\thechapter}{1em}{}
\titlespacing*{\chapter}{0pt}{18pt plus 5pt minus 5pt}{50pt plus 5pt minus 5pt}

% Proper page numbering - all pages of toc
% http://tex.stackexchange.com/questions/44974/number-on-all-pages-of-table-of-content
\makeatletter
\newenvironment{keeppage}{\let\thispagestyle=\@gobble}{}
\makeatother


%%%%%%%%%%%%%%%%%%%%%%%%%%%%%%%%%%%%%%%%%%%%%%%%
% Draftmark package commmented out
%%%%%%%%%%%%%%%%%%%%%%%%%%%%%%%%%%%%%%%%%%%%%%%%
%\usepackage[draft,allpages]{draftmark}

%\draftmarksetup{mark={\begin{minipage}{0.1\textwidth} \begin{flushright} CONFIDENTIAL \end{flushright} \end{minipage}},color=gray,grayness=0.5,scale=0.1,ycoord=-125,angle=10}
%\draftmarksetup{mark=CONFIDENTIAL,color=gray,grayness=0.5,scale=0.1,xcoord=30,ycoord=-125,angle=0}


%%%%%%%%%%%%%%%%%%%%%%%%%%%%%%%%%%%%%%%%%%%%%%%%
% Mathematics
% http://en.wikibooks.org/wiki/LaTeX/Mathematics
%%%%%%%%%%%%%%%%%%%%%%%%%%%%%%%%%%%%%%%%%%%%%%%%
% Defines new environments such as equation,
% align and split 
\usepackage{amsmath}
% Adds new math symbols
\usepackage{amssymb}

%%%%%%%%%%%%%%%%%%%%%%%%%%%%%%%%%%%%%%%%%%%%%%%%
% Bibliography
% http://en.wikibooks.org/wiki/LaTeX/Bibliography_Management
%%%%%%%%%%%%%%%%%%%%%%%%%%%%%%%%%%%%%%%%%%%%%%%%
% Add the \citep{key} command which display a
% reference as [author, year]
\usepackage{natbib}
% Appearance of the bibliography
\bibliographystyle{apalike}

%%%%%%%%%%%%%%%%%%%%%%%%%%%%%%%%%%%%%%%%%%%%%%%%
% Misc
% http://en.wikibooks.org/wiki/LaTeX/Bibliography_Management
%%%%%%%%%%%%%%%%%%%%%%%%%%%%%%%%%%%%%%%%%%%%%%%%
% Add bibliography and index to the table of
% contents
\usepackage[nottoc]{tocbibind}
% Add the command \pageref{LastPage} which refers to the
% page number of the last page
\usepackage{lastpage}

% FixMe package to insert notes
% Doc-Link: http://www.lrde.epita.fr/~didier/software/fixme.pdf
% The xkcltxp package is needed to make the fixme-package work.
% fxsetup is used to setup different parameters for the fixme-package
\usepackage{xkvltxp}
\usepackage[draft, english]{fixme}
\fxsetup{footnote,nomargin}

% tocvsec2 - set table of content depth with \settocdepth{section}
\usepackage{tocvsec2}

% Landscape pages: http://en.wikibooks.org/wiki/LaTeX/Page_Layout #Change_orientation_of_specific_part
\usepackage{pdflscape}

% package for customizing enumerate, itemize, description environments
\usepackage{enumitem}

% packages for using tikz figures
\usepackage[subrefformat=parens,labelformat=parens]{subfig}
\usepackage{tikz}
\usepackage{pgf-umlsd} % UML figures, must come before tikz-uml, to avoid layer fuck-up


\usepackage{tikz-uml} % UML figures
\usetikzlibrary{arrows,automata,decorations.pathreplacing,backgrounds,positioning,calc,shadows,shapes,backgrounds,patterns}

% package for defining acronyms. see acronyms.tex for definitions
\usepackage[printonlyused]{acronym}

% package for listing source code.
% lstset from: http://en.wikibooks.org/wiki/LaTeX/Source_Code_Listings
\usepackage{listings}
\definecolor{mygreen}{rgb}{0,0.6,0}
\definecolor{mygray}{rgb}{0.5,0.5,0.5}
\definecolor{mymauve}{rgb}{0.58,0,0.82}
\lstset{ %
  backgroundcolor=\color{white},   % choose the background color; you must add \usepackage{color} or \usepackage{xcolor}
  basicstyle=\footnotesize,        % the size of the fonts that are used for the code
  breakatwhitespace=false,         % sets if automatic breaks should only happen at whitespace
  breaklines=true,                 % sets automatic line breaking
  captionpos=b,                    % sets the caption-position to bottom
  commentstyle=\color{mygreen},    % comment style
  deletekeywords={...},            % if you want to delete keywords from the given language
  escapeinside={\%*}{*)},          % if you want to add LaTeX within your code
  extendedchars=true,              % lets you use non-ASCII characters; for 8-bits encodings only, does not work with UTF-8
  frame=single,                    % adds a frame around the code
  keywordstyle=\color{blue},       % keyword style
  language=Octave,                 % the language of the code
  morekeywords={*,...},            % if you want to add more keywords to the set
  numbers=left,                    % where to put the line-numbers; possible values are (none, left, right)
  numbersep=5pt,                   % how far the line-numbers are from the code
  numberstyle=\tiny\color{mygray}, % the style that is used for the line-numbers
  rulecolor=\color{black},         % if not set, the frame-color may be changed on line-breaks within not-black text (e.g. comments (green here))
  showspaces=false,                % show spaces everywhere adding particular underscores; it overrides 'showstringspaces'
  showstringspaces=false,          % underline spaces within strings only
  showtabs=false,                  % show tabs within strings adding particular underscores
  stepnumber=1,                    % the step between two line-numbers. If it's 1, each line will be numbered
  stringstyle=\color{mymauve},     % string literal style
  tabsize=2,                       % sets default tabsize to 2 spaces
  title=\lstname                   % show the filename of files included with \lstinputlisting; also try caption instead of title
}

% package for drawing nice folder/directory tree structures.
%\usepackage{dirtree}

%%%%%%%%%%%%%%%%%%%%%%%%%%%%%%%%%%%%%%%%%%%%%%%%
% Hyperlinks
% http://en.wikibooks.org/wiki/LaTeX/Hyperlinks
%%%%%%%%%%%%%%%%%%%%%%%%%%%%%%%%%%%%%%%%%%%%%%%%
% Enable hyperlinks and insert info into the pdf
% file. Hypperref should be loaded as one of the 
% last packages
\usepackage[hyphens]{url} % fix urls in bibliography to be wrapped when they are long.
\usepackage{hyperref}
\hypersetup{%
	%pdfpagelabels=true,% commented out, gave warning: Package hyperref Warning: Option `pdfpagelabels' has already been used,
	plainpages=false,%
	pdfauthor={--, --},%
	pdftitle={Collective data aggregation},%
	pdfsubject={Something something},%
	bookmarksnumbered=true,%
	colorlinks,%
	citecolor=black,%
	filecolor=black,%
	linkcolor=black,%
	urlcolor=black,%
	pdfstartview=FitH%
}
%%%%%%%%%%%%%%%%%%%%%%%%%%%%%%%%%%%%%%%%%%%%%%%%
% Font style
%%%%%%%%%%%%%%%%%%%%%%%%%%%%%%%%%%%%%%%%%%%%%%%%
% Enable strike through or strike out of text with \sout{TEXT}
\usepackage[normalem]{ulem}

\usepackage{multirow}

\usepackage{float}

%%%%%%%%%%%%%%%%%%%%%%%%%%%%%%%%%%%%%%%%%%%%%%%%%%
% Definitions for quotation styles:
% Ref: http://tex.stackexchange.com/questions/16964/block-quote-with-big-quotation-marks

\newcommand*\openquote{\makebox(25,-22){\scalebox{2}{``}}}
\newcommand*\closequote{\makebox(25,-22){\scalebox{2}{''}}}

% End of quotation style definitions
%%%%%%%%%%%%%%%%%%%%%%%%%%%%%%%%%%%%%%%%%%%%%%%%%%

%\newcommand*\cleartoleftpage{%
%	\cleardoublepage
%		\ifodd
%			\value{page}\hbox{}\newpage
%		\fi
%}

%\makeatletter
%\newcommand*{\cleartoleftpage}{%
%  %\cleardoublepage
%    \if@twoside
%	    \ifodd\c@page
%	      \hbox{}\newpage
%	    \fi
%  	\fi
%}
%\makeatother

% verbatim boxed text
\usepackage{fancyvrb,fancybox,calc} 
\newenvironment{code}{\VerbatimEnvironment% 
  \noindent
  %      {\columnwidth-\leftmargin-\rightmargin-2\fboxsep-2\fboxrule-4pt} 
  \begin{Sbox} 
  \begin{minipage}{\linewidth-2\fboxsep-2\fboxrule-4pt}    
  \begin{Verbatim}
}{% 
  \end{Verbatim}  
  \end{minipage}   
  \end{Sbox} 
} 

%%%%%%%%%%%%%%%%%%%%%%%%%%%%%%%%%%%%%%%%%%%%%%%%
% table setup
%%%%%%%%%%%%%%%%%%%%%%%%%%%%%%%%%%%%%%%%%%%%%%%%
% Packet to support tables across multiple pages
\usepackage{longtable}
\usepackage{colortbl}

%%%%%%%   THIS IS WHAT I ADD

% tabularx allows some features for tables
\usepackage{tabularx}
