%  A simple AAU report template.
%  2011-12-04 v. 0.1.1
%  Copyright 2010-2011 by Jesper Kjær Nielsen <jkn@es.aau.dk>
%
%  This is free software: you can redistribute it and/or modify
%  it under the terms of the GNU General Public License as published by
%  the Free Software Foundation, either version 3 of the License, or
%  (at your option) any later version.
%
%  This is distributed in the hope that it will be useful,
%  but WITHOUT ANY WARRANTY; without even the implied warranty of
%  MERCHANTABILITY or FITNESS FOR A PARTICULAR PURPOSE.  See the
%  GNU General Public License for more details.
%
%  You can find the GNU General Public License at <http://www.gnu.org/licenses/>.
%
%
%
% see, e.g., http://en.wikibooks.org/wiki/LaTeX/Customizing_LaTeX#New_commands
% for more information on how to create macros

%%%%%%%%%%%%%%%%%%%%%%%%%%%%%%%%%%%%%%%%%%%%%%%%
% Macros for the titlepage
%%%%%%%%%%%%%%%%%%%%%%%%%%%%%%%%%%%%%%%%%%%%%%%%

%Creates the aau titlepage
\newcommand{\aautitlepage}[3]{%
  {
    %set up various length
    \ifx\titlepageleftcolumnwidth\undefined
      \newlength{\titlepageleftcolumnwidth}
      \newlength{\titlepagerightcolumnwidth}
    \fi
    \setlength{\titlepageleftcolumnwidth}{0.5\textwidth-\tabcolsep}
    \setlength{\titlepagerightcolumnwidth}{\textwidth-2\tabcolsep-\titlepageleftcolumnwidth}
    %create title page
    \thispagestyle{empty}
    \noindent%
    \begin{tabular}{@{}ll@{}}
      \parbox{\titlepageleftcolumnwidth}{
        \iflanguage{danish}{%
          \includegraphics[width=\titlepageleftcolumnwidth]{figures/aau_logo_da}
        }{%
          \includegraphics[width=\titlepageleftcolumnwidth]{figures/AAU_UK_STUDENTREPORT_blue_cmyk.eps}
        }
      } &
      \parbox{\titlepagerightcolumnwidth}{\raggedleft\sf\small
        #2
      }\bigskip\\
       #1 &
      \parbox[t]{\titlepagerightcolumnwidth}{%
      \textbf{Abstract:}\bigskip\par
        \fbox{\parbox{\titlepagerightcolumnwidth-2\fboxsep-2\fboxrule}{%
          #3
        }}
      }\\
    \end{tabular}
    \vfill
    \iflanguage{danish}{%
      \noindent{\footnotesize\emph{Rapportens indhold er hemmeligt.}}
    }{%
      \noindent{\footnotesize\emph{The content of this documentation is CONFIDENTIAL.}}
    }
    \clearpage
  }
}

%Create English project info
\newcommand{\englishprojectinfo}[8]{%
  \parbox[t]{\titlepageleftcolumnwidth}{
    \textbf{Title:}\\ #1\bigskip\par
    \textbf{Theme:}\\ #2\bigskip\par
    \textbf{Project Period:}\\ #3\bigskip\par
    \textbf{Project Group:}\\ #4\bigskip\par
    \textbf{Participants:}\\ #5\bigskip\par
    \textbf{Supervisors:}\\ #6\bigskip\par
    \textbf{Copies:} #7\bigskip\par
    \textbf{Page Numbers:} \pageref{LastPage}\bigskip\par
    \textbf{Date of Completion:} #8
  }
}

%Create Danish project info
\newcommand{\danishprojectinfo}[8]{%
  \parbox[t]{\titlepageleftcolumnwidth}{
    \textbf{Titel:}\\ #1\bigskip\par
    \textbf{Tema:}\\ #2\bigskip\par
    \textbf{Projektperiode:}\\ #3\bigskip\par
    \textbf{Projektgruppe:}\\ #4\bigskip\par
    \textbf{Deltager(e):}\\ #5\bigskip\par
    \textbf{Vejleder(e):}\\ #6\bigskip\par
    \textbf{Oplagstal:} #7\bigskip\par
    \textbf{Sidetal:} \pageref{LastPage}\bigskip\par
    \textbf{Afleveringsdato:}\\ #8
  }
}


%%%%%%%%%%%%%%%%%%%%%%%%%%%%%%%%%%%%%%%%%%%%%%%%
% Macros for inserting figures
%%%%%%%%%%%%%%%%%%%%%%%%%%%%%%%%%%%%%%%%%%%%%%%%
%
% ¤¤ Image hack ¤¤ %%% Example: \figur{width}{filename}{caption}{label}
\newcommand{\figur}[4]{
		\begin{figure}[H] \centering
			\includegraphics[width=#1\textwidth]{figures/#2}
 			%\vspace{-30pt}
			\caption{#3}\label{#4}
			%\vspace{-20pt}
		\end{figure} 
}

\newcommand{\indentitem}{\setlength\itemindent{20pt}}

%New environment called req, made specifically to 
%establish requirements lists
%The syntax is \begin{req}[BOGSTAV]{TAL}
%below the requirements made by \begin{enumerate} inside the req

% Requires: \usepackage{enumitem}
\newenvironment{req}[2][]{
% first level of indentation  
  	\renewcommand{\theenumi}{#1\arabic{enumi}} % Add first parameter as prefix to enumeration
  	\renewcommand{\labelenumi}{\theenumi} % To enable label/ref support, usage: \item \label{..} Text....
% second level of indentation
 	\renewcommand{\theenumii}{\alph{enumii}}
	\renewcommand{\labelenumii}{\theenumii}
	
	\begin{enumerate}[]
	\setcounter{enumi}{#2-1} % start numbering from <2nd parameter>
}{
  \end{enumerate}
}

%%%%%%%%%%%%%%%%%%%%%%%%%%%%%%%%%%%%%%%%%%%%%%%%
% Macros for project info
%%%%%%%%%%%%%%%%%%%%%%%%%%%%%%%%%%%%%%%%%%%%%%%%
\newcommand {\ourtitle} {Title Placeholder}
\newcommand {\oursubtitle} {Subtitle Placeholer}
\newcommand {\ourtheme} {Master's thesis}
\newcommand {\ourprogramme} {Networks and Distributed System}
\newcommand {\oursemester} {10\textsuperscript{th}}
\newcommand {\ourgroup} {unknown}
