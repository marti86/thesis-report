\section{Problem formulation}
\label{sec:problemformulation}

Data traffic in networks have been increasing exponentially since the beginning of networking at the same time that new kind of services and connection requirements appear. Those factors have led to a more complex networks that cannot satisfy the needs of carriers nor users.

As explained in this introduction chapter, the current networks have several limitations to adapt to end-to-end goals requirements for the different sort of services. Determining resource allocation per class of service must be done with knowledge about traffic demands for the various traffic classes, keeping a fixed amount of bandwidth for each class, which results in a poor utilisation of resources. Nowadays methodologies are static, or need specific network equipment, which leads to a non-scalable and expensive systems. QoS is not a one-time deployment, but an ongoing, essential part of network design. Is because of that that SDN can improve significantly the management of the QoS performance, in a much easier way than the current networks.

The aim of this project is to analyse the possibilities that Software-Defined Networking bring us to develop an application able to sense the state of the network and adapt its behaviour in order to achieve a better performance and better resource utilisation of the network.

Thus, this project tries to answer the following questions: 

\quotation{\emph{How much can SDN improve the resources allocation in relation with current mechanisms?}}\\

\emph{Is it easy to implement a module in SDN that adapts the network behaviour to the present conditions guaranteeing a QoS in a scalable way?}

