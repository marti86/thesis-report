\chapter{Conclusion}
\label{chap:conclusion}

As is described in the introduction, Kamstrup is investigating the possibility of a more efficient technique for data collection, and therefore initiated a cooperation with Aalborg University. This leads to the following problem statement:

\textbf{How does a routing scheme designed for low power devices with fast data collection perform compared to Kamstrup's sequential data collection method?}

% Preanalysis
Through a pre-analysis, a bigger picture of the AMI field is achieved. This includes specifications for Kamstrup's network architecture and protocol. These two specifications are necessary to create an accurate simulation model to represent Kamstrup's sequential data collection approach. Kamstrup uses a wireless multi-hop network, where the position of the meters are fixed. A main aspect of the network devices are that the routers, routing meters and meters are battery powered, meaning that there are design choices that have to be made to ensure low power consumption. This is the reason for looking into aspects such as in-network data aggregation to minimise the number of bits transmitted. Another aspect is that the communication standard specifies that a maximum of 10 hops can be achieved using the packets, so design choices must also be made to ensure that the 10 hops are not exceeded. Kamstrup's sequential data collection approach basically performs a data collection from one meter at a time, where a concentrator waits for the previous measurement of a meter to be received before the next meter is requested. This has resulted in long data collection times that give low data granularity. In order to achieve a faster data collection time, and thereby higher data granularity, a routing scheme is designed. Therefore state-of-the-art methods are investigated to get a picture of how others have accomplished data collection in smart metering networks while taking lifetime and data collection time into account. The learning points from the state-of-the-art show that tree structures based on an optimized cost functions and combined with in-network data aggregation allow for good lifetimes. This is therefore adapted in the design of the routing scheme.

% Analysis
A simulation is performed to compare the designed routing scheme with Kamstrup's sequential data collection. In order to have realistic results real-world data are used to describe the network architecture, meaning that the placement and connectivity of the network devices in a deployed network are given. This network is deployed in Guldborgsund, Denmark. In order to compare the two approaches on this deployed network, different performance metrics are used. The main performance metric is data collection time, as it describes how fast a data collection is performed. The secondary but equally important performance metric for comparison is power consumption, as it explains if the designed routing scheme meets the requirements for low power devices. Power consumption is measured through the number of bits sent. Another set of performance metrics are used for describing the behaviour of the two approaches. These are packet filling, number of transmissions and average throughput at the concentrator. Another important part of the analysis is reliable communication as it was necessary in the designed routing scheme to include delays due to collisions. It also explains the use of retransmissions, although this is not implemented in the simulation.

% Delimitations & Design
The work of this project performs delimitations and focuses the design on three main aspects in the design of a routing scheme. These aspects are, a tree for data aggregation, in-network data aggregation and a channel implementation. The designed routing scheme is named Fast Data Collection, as it should primarily be able to perform a faster data collection compared to Kamstrup's sequential data collection. The basic design of the Fast Data Collection is to perform a setup phase consisting of generating a tree using Prim's algorithm that uses cost functions that are designed to achieve lower power consumption on a network wide scale. The cost function is based on SNR values, distances between neighbouring routing capable devices and the number of transmissions a device has performed. The task of minimising the data collection time is performed by allowing multiple transmissions to take place in the network simultaneously. The data collection round can then be initiated using broadcast requests. For in-network data aggregation, concatenation is performed using the First Fit Decreasing heuristic, which is related to the bin packing problem. This allows for a reduction in power consumption due to not sending redundant overhead. Lastly a channel is designed to handle the sending and receiving of packets and introducing the delays corresponding to collision avoidance.

% Analytical bounds
In order to get an idea of how the Fast Data Collection approach performs compared to Kamstrup's sequential data collection approach analytical bounds are derived. Only the basic principles such as broadcast request and concatenation of the Fast Data Collection approach are represented in the analytical bounds in order to make it simplistic. Through a comparison of the analytical bounds for data collection time it is seen that Kamstrup's approach has a exponential increase for increasing network size, while Fast Data Collection is linearly increasing. It can be concluded that for an increasing network size the Fast Data Collection approach becomes increasingly better compared to Kamstrup's.

% Simulation
A simulation environment is implemented in OMNeT++ and performs the simulation of the two approaches on a specific concentrator. Using specific parameters for the simulation, performance metrics are derived. It is seen that the data collection time of Fast Data Collection is reduced by 71\%, which is a clear indicator that by performing simultaneous transmissions constrained by delays due to collision avoidance gives a much faster data collection time. In relation to power consumption, it is shown that Kamstrup's approach uses 18\% more power, meaning that the Fast Data Collection approach also decreases the power consumption. It has now been shown that under the assumptions and delimitations that the Fast Data Collection approach has successfully been designed for low power devices and can perform a faster data collection. It can therefore be concluded that the designed routing scheme can significantly improve the performance of data collection compared to Kamstrup's approach.

% Future work
\section{Future Work}
Even if this project is concluded, there are certain aspects of it that can be improved and further developed. These include the presented delimitations in section \ref{sec:delimitations} as well as improvements to the already implemented methods. These abstractions can comprise subjects of future work in order to make the system more accurate, and consequently obtaining even more meaningful results from it. Going into more details about the future work, it can be divided into the following cases, which will be explained in detail later.

\begin{itemize}
\item Channel improvements
\item Cost Function improvements
\item Kamstrup routes
\item Better representation of the network
\end{itemize}

When it comes to the channel, many improvements can be introduced compared to the already implemented one. This makes it more realistic and more fitting to this specific case. One important aspect is providing a solution to the hidden node problem (as part of the collision avoidance mechanism) which affects directly the performance of the network.\\
A spacial scheduling of the simultaneous transmissions can be introduced in order to make a more efficient use of the channel by reducing the amount of collisions.\\
The current implementation assumes that no errors occur in the packets, however this is not true in a real world application. Especially when multiple transmissions are allowed to happen at the same time, the probability of bit errors increases. By taking these errors into account the data collection time could potentially increase and give a more realistic comparison of the two data collection protocols.\\
Lastly, the retransmission mechanism that was not implemented in this project should be built in to enhance the reliability of the system and allow for testing of extreme cases (for example low SNR conditions, multiple errors, etc.).

Another area that leaves space for improvement is the cost function used in this project to create the minimum spanning tree. As a next step, more parameters should be included to make a more informed choice about the edges of the tree. This increase in complexity might lead to further improvement in the metrics examined in this project.\\
Additionally, the actual routes of Kamstrup's should be used in order to get a more realistic image of the network and thereby get a better comparison of the two protocols.\\
It is also considered that a better and more accurate implementation of the simulation could be achieved by relating the packet size that a node generates with the type of the node.\\
Finally, a backup strategy (e.g. secondary routes) has to be developed in order to ensure reliability.

Having presented an outlook of the potential expansion and improvement of this work over multiple aspects, it makes sense to reflect back on the final results. The results extracted in this project make a compelling argument that the Fast Data Collection approach is worth of following and developing further.