\chapter{Analysis}
\label{chap:analysis}

\section{Description of the data provided by Kamstrup}
\label{sec:data_description}

In the scope of this project, data is provided by Kamstrup in order to have a realistic interpretation of the network. The given data contains information about two types of networks. One of the networks is located in Finland and the other one is located in Guldborgsund, Denmark. The essential difference between the two networks is the fact that the Finnish network is powered through power lines. The network located in Denmark consists of both battery-powered and power line-connected nodes, thus introducing energy constraints and is therefore the focus of this project.
%Towards this direction a decision was made to focus the scope of this project on the Danish network, and furthermore work under the assumption that all nodes are battery-powered as to examine the worst case scenario that can occur in this type of network.

More into detail, the aforementioned data comes in the form of several files, most of them of .csv format but also .kml (format supported from Google Maps - geovisualization of the networks). The file named "Installations" (.csv) contains information about all the nodes in the network such as name of the node, longitude, latitude, etc. The structure of the Installations file is seen in table \ref{tab:installations}. The three possible roles for a node are Router, Vand (Water) and Varme (Heating).
% Matrix for Installations.csv
\begin{table}[H]
\begin{center}
    \begin{tabular}{ | l | l | l | l |}
    \hline
    Name & Longitude & Latitude & Role \\ \hline
    9849 & 11.65226 & 54.774 & Router \\ \hline
    9850 & 11.6117 & 54.76089 & Router \\ \hline
    9851 & 12.4092 & 54.89005 & Router \\ \hline
    7033271 & 11.91317 & 54.76061 & Vand \\ \hline
    7035153 & 11.87783 & 54.75621 & Varme \\
    \hline
    \end{tabular}
\end{center}
\caption{Sample data of useful information from the Installations file}
\label{tab:installations}
\end{table}

A set of files named "LocalLists\_*" (.csv), where * is the id of a node with routing capabilities, consist of information on connectivity of either a router or a meter with implemented routing module. The structure of a LocalLists file is seen in table \ref{tab:locallists}. These files are only associated with nodes that have routing capabilities and provide information about the neighbourhood of the specific node through its respective SNR values. Meters do not have LocalLists, but their connectivity to routers can be derived from the LocalLists of the routers.

\begin{table}[H]
\begin{center}
    \begin{tabular}{ | l | l | l | l |}
    \hline
    From node & To node & SNR\\ \hline
    9597 & 7035445 & 29 dB \\ \hline
    9597 & 7035464 & 27 dB \\ \hline
    9597 & 7035408 & 28 dB \\
    \hline
    \end{tabular}
\end{center}
\caption{Sample data of useful information from a LocalLists file}
\label{tab:locallists}
\end{table}

\figur{0.8}{google-earth}{Screenshot of Google Earth environment representing the metering network in Guldborgsund, Denmark. Each multi-coloured dot is a router/meter, while a yellow landmark is a concentrator.}{fig:google-earth}

A file that allows the geovisualization of the network through the Google Maps services is also provided. Even if this file is not directly applied in the simulation it helps so one can visually understand the layout of the network and the distribution of the nodes, be it concentrators, routers or meters. The AMI in Guldborgsund, Denmark, is seen in figure \ref{fig:google-earth}, and gives a graphical overview of the distribution of the nodes. A matlab plot of the same region is seen in figure \ref{fig:guldborgsund}, where the different types of nodes are represented. It can also be seen that in some households are both meters and routing meters (meters with routing modules) in the same location. The distribution of type of nodes for the same area is seen in table \ref{tab:amount_of_nodes}. For this area, it is seen that here are almost as many routing meters as meters and almost no dedicated routers.

\begin{table}[H]
\begin{center}
    \begin{tabular}{ | l | l | l | l |}
    \hline
    Type & Amount & Percentage \\ \hline
    Meter & 4556 & $\approx$ 51 \% \\ \hline
    Dedicated Router & 67 & $\approx$ 0.7 \% \\ \hline
    Routing meter & 4294 & $\approx$ 48 \% \\ \hline
    Concentrator & 31 & $\approx$ 0.3 \% \\
    \hline
    \end{tabular}
\end{center}
\caption{The amount of different types of nodes within in Guldborgsund, Denmark}
\label{tab:amount_of_nodes}
\end{table}

\figur{0.8}{guldborgsund}{Plot of the locations of different types of nodes in Guldborgsund, Denmark}{fig:guldborgsund}

A description of the routes between a concentrator and the meters is provided. This is detailed in a file named NetLists. It describes a parent, a cild and the number of remaining hops to the concentrator. The project focuses on generating routes in the network, while keeping the assigned meter-to-router associations as based on the values provided in the LocalLists.

The specific data provided by Kamstrup will provide a realistic system description to the simulated network by utilizing real-world positions and SNR values.

In order to create the simulation based on the previously mentioned data it is imperative that specific fields need to be extracted from the data files and used inside the simulation environment. The data will mainly be used for the creation of the nodes in the simulation environment and for generation of routes. %Data that would make sense and we want to include in our simulation are the longitude and latitude for each node along with their unique id and their role in the network (concentrator, router, meter). These data is of key importance during the creation of the nodes and the network. It allows for accurate positioning of the nodes into the simulation "playground" while being able to differentiate them through their unique IDs. An adjacency matrix, constructed from the local lists provided by Kamstrup, containing the SNR values for each pair of node (thus depicting each node's "neighbourhood") would also need to be included in the simulation. This matrix provides a clear and correct image on how the nodes should be connected in the network. This is because we seek the maximum SNR value to have the strongest connection between two nodes.
% TODO - Close the subchapter in a less abruptly

\subsection{Distribution of data}
\label{sec:data_distribution}
An analysis of the distributions of the SNR values and distances between network devices has been performed. A single concentrator, with ID 10053, placed at latitude $11.8811^{\circ}$ and longitude $54.7613^{\circ}$ is the focus of this analysis. A network within a radius of $250m$ of the concentrator is analysed and is seen in figure \ref{fig:concentrator10053}.

\figur{0.8}{concentrator10053}{A concentrator with network devices within a range of $250$m plotted using latitudes and longitudes.}{fig:concentrator10053}

Figure \ref{fig:snr_dist} shows the distribution of SNR values. It is seen that there is a large amount of the distribution that does not have a sufficient link quality above $18dB$. The amount of SNR values under $18dB$ corresponds to $42.26\%$ of the dataset, meaning that $42.26\%$ of the connections do not allow for data exchanges.


\figur{0.75}{snr_dist}{Distribution of SNR values with bin size of approx. $2.3dB$}{fig:snr_dist}

Figure \ref{fig:distance_dist} shows the distribution of distances that each network device has to all its neighbours. It is also seen that $8.84\%$ of the devices have a distance of $0$m, when using the differences in latitudes and longitudes. This means that there are cases where there are multiple devices placed in the same household. The cases where this arises is for collection of multiple types of consumption. In this case there are up to three devices in the same household.

\figur{0.75}{distance_dist}{Distribution of distances with bin size of approx. $0.7m$}{fig:distance_dist}














\section{Simulation}
\label{sec:simulation}

% why - what do we expect to get out of this simulation?
% move expectations up -> we expect to get parameters about the best way to get a get fast data collection}

From a practical point of view, simulations are more often used to predict a system's behaviour and can be used for conditions that are difficult or even impossible through testbeds due to scalability constraints. When it comes to large networks, the most efficient approach of measuring their performance is to create a realistic simulation that approximates the system's expected behaviour. The simulation approach helps us obtain a better and directly comparable view of the system's performance.\\
Moreover, it is generally less costly than experimenting in a testbed. The only existing costs are the simulation software cost and the work hours that will be spent to create the simulation model. Under real, non-simulated, conditions the downsides of testing include longer run-times, a longer set-up phase and hardware costs. The simulation approach offers repeatability, through which a large number of scenarios can be tested and compared, examining the exact same system under different input or implemented routing schemes. Adding to that, by increasing the complexity of the simulation model a better representation of the system can be achieved, leading to simulation results closer to results from the real world. Lastly, simulation can offer a shorter run-time compared to a testbed, allowing fast results when changing parameters between tests. An advantage to simulation is the ability to investigate if a system is future-proof, meaning for example that the network size can be scaled higher than what is seen in a real world environment. Scaling a network is not easily achievable for testbeds since it will require many hardware components. The main advantages of simulating are presented below:

\begin{enumerate}
\item Approximate depiction of reality
\item Insightful system evaluation
\item Future proofing of the system
\end{enumerate}

\subsection{OMNeT++ : What and Why}
It was chosen to use the OMNeT++ simulation tool \citep{omnet}, because it has a good performance for large networks and some group members have previous experience working with the tool. Furthermore OMNeT++ offers a variety of frameworks. A general description of the OMNeT++ network simulator is provided below.

OMNeT++ is a network simulator that can support several kinds of networks, including wired, wireless, satellite, etc. Its functionalities can also be extended through specific frameworks that improve the detail and functionalities of specific network types. Such examples are the Mixim framework, which provides mobility features, and the INET framework that supplements the wireless network simulation palette. OMNeT++ provides its own data format (.NED files) for the creation and operation of the network, however at the base level all the functionalities are built using the C++ programming language that allows for modularity and customization down to the most basic components; that is the user being able to create his own submodules to implement specific functionalities that exist or fit in his specific case.

However in the cases examined in this project, since they are quite different from what is usually implemented, most of the work was done from scratch utilizing OMNeT++'s simulation code support and programming the necessary modules and functionalities in C++. It is feasible at this point to make an analytic reference to the components of the network that was created through OMNeT++ and what is the significance and role of each. The network consists of four kinds of nodes, The concentrator (this is the initiator of the communication and also the recipient of all the aggregated data), the routers (their tasks are to route the packets from the concentrator to the meters and vice versa and will in the designed approach also perform in-network data aggregation), the meters that have routing modules (meters that additionally are able to route traffic and also perform in-network data aggregation) and lastly the meters (meters generate measurements of resource consumption, and they send their data towards the concentrator).

Regarding the simulation implementation, Kamstrup provide relevant data from their network so a connected graph can be constructed to represent the network. Kamstrup's data will provide information about the connectivity and location of devices in the network.
The main scope of this project will be to improve the data collection process of the network, by implementing a routing scheme, as explained in the problem formulation in section \ref{sec:problemformulation}. Towards that end, the task is to implement a set of protocols and compare their performance in the imported network. One of the protocols will be Kamstrup's sequential data collection and the other one will be a designed routing scheme that will be inspired by state-of-the-art solutions from the Automated Meter Infrastructure (AMI) field of research.

In AMI systems such as Kamstrup's, end goal is fast and energy-efficient data collection. The simulation approach allows for testing and tweaking of several parameters that affect this specific objective. The output of the simulation regarding the performance of the network should allow us to understand what parameters of the network and under which specific setup, have greater impact towards a fast and reliable data collection.

A set of goals and certain expectations from the simulation approach are now defined. One of the expectations, and maybe the most important, is to find parameters that optimize the data collection task. The term `optimise' is used to mainly describe the focus of the simulation in regard to the data collection aspect that we wish to optimise. This can be, data collection  time, throughput, power consumption, etc.



\section{Performance metrics}
\label{sec:performance_metrics}

This section will describe the different performance metrics that are to be derived from the simulation. There will be two types of performance metrics. The first set of performance metrics is used for comparison between Kamstrup's sequential data collection and the designed routing scheme. These are related to the problem formulation (see section 
\ref{sec:problemformulation}), describing that this project will seek to design a routing scheme that can perform fast data collection for low power devices. This implies using two performance metrics, data collection time and power consumption. The other set of performance metrics are used for describing the behaviour of the two approaches. These include packet filling (a data aggregation metric), number of transmissions and average throughput at the concentrator.

\subsubsection{Data collection time}

The main scope of this project is heavily related to Kamstrup and their data aggregation scheme. What became clear from the first meeting with Kamstrup's contact person is the fact that data precision, meaning often updates, and the ways this can achieve it is a driving factor in this kind of networks and industry. It is the developmental drive as resources providers are interested in more precise data, more often in order to have a clearer view of customer consumption and usage. It is also the most prominent way towards implementing a smart grid. Figure \ref{fig:data_collection_terms} shows the different terms related to a data collection process.

\figur{0.7}{data_collection_terms}{Different terms related to a data collection process}{fig:data_collection_terms}

The data precision is given through a data collection interval, and describes how often a data collection round is started. A small data collection interval is what defines a greater precision in the data. In order to achieve a smaller data collection interval it is necessary to perform a fast data collection and therefore achieve a smaller data collection time. A data collection time is the time it takes between the first data request (initiated from the concentrator) until the last data response (received by the concentrator). The only constraint that exists between the data collection time and the interval is a logical one, that the data collection interval should be greater or equal to the any data collection time. 

The data collection time is deemed to most accurate way to compare the performance of Kamstrup's sequential data collection and the designed routing scheme, since it shows if a faster data collection has been achieved.

\subsubsection{Power consumption}
% secondary performance metric for comparison -> low power devices

Power consumption is an important metric for comparison as the network used during this project is assumed to work solely on battery. This performance metric is used to attain that the designed routing scheme has been designed for low power devices. It should achieve a lower or keep the same power consumption as in Kamstrup's sequential data collection approach. Power consumption is measured as the number transmitted bits, since no information is provided on the battery consumption on the network devices.

Using the number of transmitted bits as a measurement of power consumption it is only logical to compare the relative power consumption between the different approaches. For measuring power consumption the overhead in sent packets will have a size of 23 bytes and requests will be assumed to all have a size of 10 bytes. It could also be possible to modify the power consumption scheme directly from the implemented protocol in OMNeT++ if the need arises.

\subsubsection{Packet filling}
Packet filling is a performance metric related to describing the behaviour of the designed routing scheme's in-network data aggregation. Packet filling is defined as the usage in percent of the maximum packet size (250 Bytes), when using the concatenation method. Power consumption and packet filling metrics are related to each other. If a high packet filling is achieved a lower power consumption will be achieved. This is related to redundant overhead being avoided and therefore less bits being transmitted.

\subsubsection{Number of transmissions}
The number of transmissions happening in the network throughout a data collection round is also deemed a very important metric to acquire an accurate image of the behaviour of the two approaches. The number of transmissions can show the activity throughout a data collection round. It shows the performance of the concatenation method, as fewer transmissions mean better concatenation. The number of transmissions can also be used to give a rough estimate of its power consumption. It also possible to show the progression of the data collection in terms of (simultaneous) transmissions.

\subsubsection{Average throughput}
The average throughput is a performance metric describing the utilisation of the bandwidth at the one-hop links to the concentrator. It is calculated by taking the summation of the received packet sizes over the data collection time.


\section{Reliable communication}
\label{sec:reliable_communication}
Reliability is defined as the ability of a certain system to steadily operate according to predefined specifications. The concept of reliability is a cornerstone in the design of a communication system. The objective is the guaranteed transmission of a message in relation to both data and connection integrity. Reliability is a generic concept for both wired and wireless communications, but of major importance. Especially in the latter where the medium is one (usually the air) and limited (to the frequency bands), reliability -along with the network planning and system design- will severely affect the overall system's performance. Reliability mechanisms are applied on different levels depending on the imperfections they try to suppress. These mechanisms usually want to:

\begin{itemize}
\item Deliver data without bit errors
\item Provide link reliability
\item Share the common medium (band) with no collisions
\end{itemize}

% Why is reliability relevant to our case?
The relevance of reliability mechanisms to this project are investigated. After investigating more thoroughly, it is concluded that there is correlation to the project, which is based on the last two bullets and resides in two distinct factors. Firstly, the freedom these mechanisms provide when transmitting frames over the channel by acting as a level of abstraction. Secondly, the additional delays introduced as result of these mechanisms. This does not mean that the overall system's performance is going to be degraded due to these delays, but on a link basis (node to node) delays will be introduced to keep the channel clean while neighbouring nodes want to transmit their data.

% Kamstrup approach
Kamstrup's system is already there, up and running, meaning that the aforementioned challenges have already been confronted. The first problem of delivering the data packets without errors is solved by adding a CRC (Cyclic Redundancy Check) at the end of the packet. This way the corrupted packets are identified and the packet is requested again. The second issue of providing reliable links is solved on a first level by adding dedicated routers to act as repeaters and provide paths to unreachable nodes and on a second level by choosing routes with strong SNR and high reliability of link. The third issue concerning the sharing of the licensed band is solved by simply not sharing it, meaning that the data is requested sequentially from each node so that only one node in the network can transmit at each moment. Note that data collections can be performed in parallel in different subnets (subnet: a network consisting of a concentrator and all the installations associated to it). %This approach has major differences from the one that is implemented in the simulation model. Moreover, it is chosen to abstract from those mechanisms that were not directly affecting the outcome of comparison between the two models.

An additional mechanism chosen and implemented by Kamstrup is to create secondary routes for each installation. When a problem at the primary route is detected (i.e. bad link quality between two routing modules on the path) the concentrator requests the data packets to be delivered using the secondary route. In this project we do not consider changes in the environment, which results in static links and makes such a functionality irrelevant, thus giving us the possibility to abstract from it.

On the contrary, the fact that popular methods of providing reliability, such as Forward Error Correction, CSMA/CA, etc., would not be available was clear from the beginning. That is due to the hardware constraints of the given existing system. Despite this fact, it is chosen to provide an implementation based on an approach that would not be delimited by it.

In summary, this project does not try to address the reliability issues, but wants to model the effect (delays) of the most relevant reliability mechanisms on the system. The effective channel sharing is considered to be the most challenging task and the one that can set the boundaries to the performance of this specific system.

In the following subsections a description of techniques that a considered for sharing the common channel.

\subsection{Collision avoidance}

In a network where multiple nodes operate, sharing a common medium to interchange messages, collisions are bound to occur when no mechanisms or rules apply on the transmissions. A collision is the corruption of the transmitted frames when more than one node accesses the common medium at the same time. Collision avoidance mechanisms ensure to prevent collisions without degrading the medium utilization.

Telecommunications distinguish the concept of `sensing the carrier' (CSMA) in two different categories according the actions they perform after sending it. The first detects the collisions (CSMA/CD) and retransmits the data after a random back-off interval. This is more appropriate and thus widely used in wired networks. The second one attempts to avoid collisions (CSMA/CA) either by sensing the channel or by receiving a clear-to-send message from the receiver node. This approach performs well on wireless networks. The clear-to-send message solves the `Hidden Node Problem' that is described in the figure below.

\figur{0.3}{hidden_node}{The hidden node problem in wireless communications: Node A and C cannot hear each other so they try to send to B at the same time. Inside the area B is in the collision occurs \citep{hiddennodefig}}{fig:hidden_node}

%Fast data collection protocol uses a striped down mechanism of collision avoidance. Our decision for implementing such functionalities on certain types of our nodes was motivated by the necessity of having multiple transmissions in order to realize our protocol. The most relevant outcome of this implementation for our project is providing determinant delays as part of the model. In the case where these delays were not introduced

In this project, among the aforementioned techniques, we use carrier sensing and we introduce a back-off after the detection of another transmission from the neighbours. This means that the nodes of the network that need to send data, listen to the channel and when the channel is detected as idle they take it in order to send their data.  Moreover, there are other aspects to consider like the sleep/wakeup window implemented on lower layers. According to this, nodes can only have a specific time window in order to perform any operations or transmissions.\\
Since the scope of this project is to introduce a routing scheme, collision avoidance is utilized in order to provide the determinant delays of such a mechanism to our model but at the same time to provide a basis for implementing more powerful routing algorithms. One of the objectives of this project is to also improve the current data collection method used by Kamstrup. With this as a given, one can understand that medium access control methods are key to this case. This is what the access with collision avoidance control does in this project in a rather simplistic way.

\subsection{Retransmission}

The retransmission mechanism guarantees that all the messages will reach the end destination, in this case the concentrator. It can be performed on a hop-by-hop basis or on an end-to-end basis by performing a CRC check. In the case where the packet is received with no errors an Acknowledgment (ACK) packet is sent back. When the sender does not receive an ACK within a time limit, it attempts to send the data again. This is a generic way reliable transmission protocols (e.g. Transmission Control Protocol) operate in order to assure data integrity. 

In the designed approach, this mechanism will operate differently in order to comply with the energy constraints. The nodes that act as simple meters only respond to the router they are assigned to, when they are asked to do so. The nodes that operate as routers check the packet correctness and trigger a retransmission from the specific node whenever the packet is found to be corrupt. Likewise, when there is a corrupt packet received when transmitting from one router to another router, a retransmission will also occur. Following this approach, slightly increases the processing needs on the routing module, but saves a lot of energy by minimizing the transmissions.

The reasoning behind the need of such a functionality, which is fairly simpler and less effective than the collision avoidance, relies on the reasons mentioned below:

\begin{itemize}
\item Simple meter's capabilities are bounded to a great extent by their hardware.
\item Listening to the channel cannot be justified for meters, when only one packet is sent from them and packets drops occur rarely.
\end{itemize}
% Bullet explanation
The nodes with routing modules are considered to be more sophisticated. The nodes with no routing capabilities, on the other hand, are treated as the simplest network devices. It is assumed that on these meters no radical changes can be performed, since they are constrained even more by their hardware. Moreover, the objective of such devices is limited to transmitting their own data packets. This means that it wouldn't be so practical sensing the channel just for one packet transmission.


% % % % % % % % % % % % % % % % % % % % % % % % %
%As mentioned before, the retransmission mechanism has a fairly simple implementation as it retransmits the packet (once) to its intended destination. This way we ensure that the concentrator, which is ultimately the destination of every message, receives the messages from all the nodes. It can be made clear from the preceding text that retransmission techniques mostly operate on an end-to-end level where the delay criteria tend to scale up and become a more critical matter to the performance of the network.

%Additionally to the minimisation of delay of the network via retransmissions, we also want to evaluate the technique itself as we are heavily interested in the power consumption of the nodes and the network in overall. As we want to minimise the number of transmissions from the nodes, this coherently affects the retransmissions as well. This way we also conserve network resources and battery power (through processing power minimisation).
% Are we going to set a specific retransmission time-out? - We are but it is going to go in the implementation section
%Ending this subsection, the purpose of the retransmission mechanism is highlighted. It implements the redundancy we need, under certain constraints, in order to ensure reliable and fast data aggregation of all the messages in the system to the concentrator. The retransmission approach is deemed to be the better fit regarding the scope of this project since there are physical limitations on both the application layer of the nodes as well as in the hardware that limit the extensive use of more advanced control techniques e.g. Forward Error Correction (FEC), CSMA/CA, etc.


\section{Conclusion}
This section has analysed various topics of interest starting with the data provided by Kamstrup. The provided data allows a simulation to be performed on data from a real network, so results can show how the approaches can perform in the real world. It has also been explained why a simulation is performed. This is due to simulations offering a way to perform a performance analysis with small execution times that allow for big networks. The set of two types of performance metrics are presented. Those used for comparison of Kamstrup's sequential data collection and the designed routing scheme and those used to describe the behaviour of the two approaches. Lastly reliable communication is analysed to give a picture of what aspects are necessary to consider for reliable communication and if multiple transmissions occurs. With the goal of acquiring realistic delays when performing multiple transmissions. The method of collision avoidance was chosen to be a functionality of the nodes with routing capabilities, whereas the simple retransmission method would be applied on the simple meters.