\section{Performance metrics}
\label{sec:performance_metrics}

This section will describe the different performance metrics that are to be derived from the simulation. There will be two types of performance metrics. The first set of performance metrics is used for comparison between Kamstrup's sequential data collection and the designed routing scheme. These are related to the problem formulation (see section 
\ref{sec:problemformulation}), describing that this project will seek to design a routing scheme that can perform fast data collection for low power devices. This implies using two performance metrics, data collection time and power consumption. The other set of performance metrics are used for describing the behaviour of the two approaches. These include packet filling (a data aggregation metric), number of transmissions and average throughput at the concentrator.

\subsubsection{Data collection time}

The main scope of this project is heavily related to Kamstrup and their data aggregation scheme. What became clear from the first meeting with Kamstrup's contact person is the fact that data precision, meaning often updates, and the ways this can achieve it is a driving factor in this kind of networks and industry. It is the developmental drive as resources providers are interested in more precise data, more often in order to have a clearer view of customer consumption and usage. It is also the most prominent way towards implementing a smart grid. Figure \ref{fig:data_collection_terms} shows the different terms related to a data collection process.

\figur{0.7}{data_collection_terms}{Different terms related to a data collection process}{fig:data_collection_terms}

The data precision is given through a data collection interval, and describes how often a data collection round is started. A small data collection interval is what defines a greater precision in the data. In order to achieve a smaller data collection interval it is necessary to perform a fast data collection and therefore achieve a smaller data collection time. A data collection time is the time it takes between the first data request (initiated from the concentrator) until the last data response (received by the concentrator). The only constraint that exists between the data collection time and the interval is a logical one, that the data collection interval should be greater or equal to the any data collection time. 

The data collection time is deemed to most accurate way to compare the performance of Kamstrup's sequential data collection and the designed routing scheme, since it shows if a faster data collection has been achieved.

\subsubsection{Power consumption}
% secondary performance metric for comparison -> low power devices

Power consumption is an important metric for comparison as the network used during this project is assumed to work solely on battery. This performance metric is used to attain that the designed routing scheme has been designed for low power devices. It should achieve a lower or keep the same power consumption as in Kamstrup's sequential data collection approach. Power consumption is measured as the number transmitted bits, since no information is provided on the battery consumption on the network devices.

Using the number of transmitted bits as a measurement of power consumption it is only logical to compare the relative power consumption between the different approaches. For measuring power consumption the overhead in sent packets will have a size of 23 bytes and requests will be assumed to all have a size of 10 bytes. It could also be possible to modify the power consumption scheme directly from the implemented protocol in OMNeT++ if the need arises.

\subsubsection{Packet filling}
Packet filling is a performance metric related to describing the behaviour of the designed routing scheme's in-network data aggregation. Packet filling is defined as the usage in percent of the maximum packet size (250 Bytes), when using the concatenation method. Power consumption and packet filling metrics are related to each other. If a high packet filling is achieved a lower power consumption will be achieved. This is related to redundant overhead being avoided and therefore less bits being transmitted.

\subsubsection{Number of transmissions}
The number of transmissions happening in the network throughout a data collection round is also deemed a very important metric to acquire an accurate image of the behaviour of the two approaches. The number of transmissions can show the activity throughout a data collection round. It shows the performance of the concatenation method, as fewer transmissions mean better concatenation. The number of transmissions can also be used to give a rough estimate of its power consumption. It also possible to show the progression of the data collection in terms of (simultaneous) transmissions.

\subsubsection{Average throughput}
The average throughput is a performance metric describing the utilisation of the bandwidth at the one-hop links to the concentrator. It is calculated by taking the summation of the received packet sizes over the data collection time.

