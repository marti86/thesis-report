\section{Constraints}
\label{sec:constraints}
% TODO:
% - The constraints should be used for delimitation and a description of the conrete problem formulation.
% add Delimitation section

Context-Aware Networks (CAN) are a special category and facilitate a specific set of services, one of them being Service-migration services. However, in order to be able to compare the different scenarios with each other we have to understand their environment. As environment we define the number and type of components and entities participating in the network as well as network topologies and mechanisms introduced to the network capable of affecting its performance. Most of the constraints were extracted through self-study and discussions with the supervisor, Rasmus L. Olsen. All the constraints are not part of the focus of this project and this fact will be discussed further on in the Delimitations section.

\subsection{Network architecture}
\label{net_arch}
There are multiple configurations that can enable a service migration in a context-aware environment. However to fulfil the scope of this project certain assumptions and constraints will be applied to limit the number of configurations that will be examined.

\begin{itemize}
\item Two or more service-migration enabled devices.
\item Limited triggering methods for the migration procedure.
\item All devices communicate through a common interface.
\item The device-discovery protocol is one.
\end{itemize}

While there are many aspects that the project can focus on, it has become clear early on the project that among the parameters that heavily affect the performance of this service are \begin{inparaenum}[\itshape a\upshape)]
\item the implemented network topology;
\item the service migration triggers; and
\item the existing configurations available to the migration service.
\end{inparaenum} as more will be explained on a later chapter. Below the network architectures that will be further examined in this project will be presented and more in-depth inspected.

\subsubsection{Centralised Architecture}
In the centralised architecture as the name dictates, there is a central authority that takes care of all the communication tasks between the application that wants to migrate and the available configurations inside the network that it can migrate to.

\figur{0.9}{prean-centralised.png}{A Centralised Architecture}{fig:pa-centralised}

From the schematic, the thing that can be derived is that the device called "Service Migration Manager" handles all the load of the migration procedure; from the migration request from the APP-side (where the migration request is based on a specific trigger), to querying all the available components, receiving their statuses and ultimately making a decision about the migration procedure. The centralised approach might be the one that conserves the most energy since the whole procedure of migration of an application is orchestrated by the intelligent unit, "Service Migration Manager".

\subsubsection{Decentralised Architecture}
In the complete opposite direction of the centralised architecture, there is the decentralised one which lacks the support of an "intelligent unit" in contrast to the previous case.

\figur{0.9}{prean-decentralised.png}{A Decentralised Architecture}{fig:pa-decentralised}

In the decentralised approach every step of the migration procedure; from component status query to the final migration decision, is handled by the APP-side. This might present with certain problems to power consumption and the triggering methods but it maintains the network and its functionalities, to a certain extent, even in the absence of a central authority. Another case in the decentralised scenario that should be mentioned is the case where an access point exists in the network setup in place of the Service Migration Manager but lacks any form of intelligence, acting only as an extender for the range of the device that wants to migrate and communicates only through a wireless interface. The decentralised approach is expected to be the most energy consuming for the migrating application since all the tasks are orchestrated by the application itself.

\subsubsection{Hybrid Architecture}
The hybrid approach combines the principals of the two, previously mentioned, architectures.

\figur{0.9}{prean-hybrid.png}{A Hybrid Architecture}{fig:pa-hybrid}

In this approach the communication task during a migration procedure can be split into two different stages. During the first stage the application that has initiated a migration procedure, based on a trigger, queries the Migration Service Manager for any or all of the available configurations depending on the migration task. The Service Migration Manager maintains a state for all the available connected to him (through Wi-Fi or Ethernet) and which he communicates to the application that wants to migrate. Upon reception of the available configurations in the network and their statuses, the application itself can make a decision for the migration procedure and proceed to contact specific devices. This way a lot of time is saved from the discovery process and the central coordination of the communication. However the migrating device is presented witha a  higher load because it still holds the final decision for the migration.



