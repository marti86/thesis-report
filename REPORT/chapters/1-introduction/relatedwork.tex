\section{Related Work}
\label{sec:related}

There is plenty of papers about traffic engineering \cite{rfc3272} trying to adapt the routing rules to the network conditions with the aim of taking the maxim profit of the available resources, and avoid traffic congestion. One of the ways to readjust the networks is using OSPF, as described in \cite{rfc3630}, by adjusting the state of the links thus the traffic can be adjusted to the current conditions .In \cite{ospfW} they compare OSPF against MPLS. trying several ways to adjust the weights setting to achieve a performance closer to the optimal in MPLS routing, but getting as a result that MPLS can achieve better performance. However, they also mention some advantages of OSPF in front of MPLS, since it is much simpler due that the routing is completely determined by one weight of each link, what avoids making decision per each source-destination pair. Also, if a link fails, the weights on the remaining links immediately determines the new routing.\\

One of the features of MPLS is its flexibility to adapt 