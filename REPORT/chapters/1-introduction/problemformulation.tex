\section{Problem formulation}
\label{sec:problemformulation}

Data traffic in networks have been increasing exponentially since the beginning of networking at the same time that new kind of services and connection requirements appear. Those factors have led to a more complex networks that cannot satisfy the needs of carriers nor users.

As explained in this introduction chapter, the current networks have several limitations to adapt to end-to-end goals requirements for the different sort of services without wasting the existing resources. 

Determining resource allocation per class of service must be done with knowledge about traffic demands for the various traffic classes, keeping a fixed amount of bandwidth for each class, which results in a poor utilisation of resources. However, the traffic that the network has to carry change constantly in an unpredictable way.

Nowadays methodologies are static, or need specific network equipment, which leads to a non-scalable and expensive systems. A new centralised control plane paradigm approach, with a global view of every element in the network, can improve significantly the management of the route for every single flow, taking in account the characteristics of those in terms of bandwidth utilisation or other parameters, to accomplish a much better utilisation  of the resources.

The aim of this project is to analyse the possibilities that Software-Defined Networking bring us to develop an application able to sense the state of the network and adapt its behaviour in order to achieve a better performance and better resource utilisation of the network.

However, resource utilisation is not the unique important factor. In order to guarantee a certain quality for the carried traffic, It's crucial to keep some parameters, such as delay or jitter, within a bounded limits.  

%Thus, this project purpose an algorithm applied to Software-Defined Networking to manage the route per each flow in order to take the maximum advantage of the available bandwidth in the network.

Thus, this project tries to assess the possibilities brought by SDN to manage the route per each flow in order to take the maximum advantage of the available bandwidth in the network, and evaluate the following points:

\begin{enumerate}
\item Improvement in terms of resources utilisation.
\item Impact of rerouting the flows.
	\begin{itemize}
	\item Delay
	\item Jitter
	\item Packet losses
	\end{itemize}
\end{enumerate}



%\quotation{\emph{How much can SDN improve the resources allocation in relation with current mechanisms?}}\\

%\emph{Is it possible to use }

