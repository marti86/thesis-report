\section{Conclusion}
\label{sec:design:conclusion}

This chapter describes the design of the system and its components. This includes physical parts like the network devices and theory like the minimum spanning tree or the in-network data aggregation. The channel can be seen as both, a physical and software part.

Kamstrup's approach has been described in detail and compared to the designed protocol named Fast Data Collection. The differences are in when requests and responses occur. While Kamstrup is currently using a sequential approach (see section \ref{sec:system_design:kamstrup}), meaning that one meter at time is requested, the Fast Data Collection sends out only a single request message and waits for responses (section \ref{sec:system_design:fdc}). Since in Kamstrup's sequential approach it is very unlikely that collisions occur, the Fast Data Collection had to use methods to avoid collisions in order to operate properly. The solution for that is the implementation of a collision avoidance. This technique is important for good functionality but not for the protocol itself. This way all devices are implemented in a two-layer structure, including an application layer and network interface card (NIC) implementing the collision avoidance (section \ref{sec:devices}). It can be shown that the NIC-layer will be the same for all devices, except for simple meters, because it does not have collision avoidance. The application layers change between devices as well as between protocols. This is obvious because the different devices perform different actions in each data collection scheme.

The initial request sent in the Fast Data Collection protocol is performed by a broadcast, which will be re-broadcasted by every network device, except the simple meters, exactly one time. This will assure that every node capable of receiving broadcasts has received it and no flooding of broadcast messages happens. Every device that has simple meters associated to it has to request the meters sequentially. This way collision will be reduced even when they do not implement collision avoidance. Another feature of this technique is that simple meters can operate as before without a need to change the firmware, what would cause an enormous effort due to the number of meters already installed. The response must than be sent towards the concentrator, which means that a route has to be defined. This is done by turning the network into a tree structure. A minimum spanning tree has been chosen in order to find the best edges in the network. ``Best'' means in this case an edge that has a good SNR-value, a length over 80m and only few transmissions had been sent over this edge before. These three thoughts are taken into account by defining a cost function as described in section \ref{sec:cost_function}. The minimum spanning tree itself will be computed by the well known algorithm of prim described in section \ref{sec:prim}. The information about the tree, meaning the direct parent and children must be spread into the network. Therefore a setup phase had been introduced.

The tree structure gives a lot of advantages to the Fast Data Collection. One basic advantage is that the number of nodes will be reduced by moving upwards closer to the concentrator. Instead of sending each packet individually a concatenation will be performed at each parent node. This allows for a reduction of energy consumption of the network devices. A major problem is the question how the packages should be concatenated in order to reduce the total number of packages. An answer can be found by looking into the bin packing problem and definition of a heuristic seen in section \ref{sec:bin_backing_problem} that will solve the problem in polynomial time. Tests have been run using the developed heuristic and results had been found to be good enough to apply it.

Since collisions are possible in the Fast Data Collection protocol, it is necessary to define a common channel to detect collisions and perform delays due to sending and back off due to the channel being occupied. This is essential to the simulation because otherwise the extracted results or not comparable. The channel is described in section \ref{sec:channel}.