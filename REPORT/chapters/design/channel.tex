\section{Channel}
\label{sec:channel}

A channel implementation to handle the transmissions over the medium was found to be necessary for the Fast Data Collection protocol. This was deemed necessary based on the fact that with this protocol data packets are requested simultaneously from multiple meters in the network. The meters instantaneously start to propagate data frames into the air. This results in collisions that need to be avoided; while this effort for avoidance causes delays as described in section \ref{sec:reliable_communication}, `Reliable communication'.

%--Trying to model a static instance of the channel Kamstrup's network devices operate in. This instance is acquired by the local lists.
%The main goal when designing a wireless channel is to apply a propagation model such as Friis, Egli, Young, etc. in order to simulate the path loss of each node.\\
The given network can be seen as static and is delimited not to have any changes in the environment. Moreover, Kamstrup provides the SNR values describing what any routing module can ``hear''. As a result of the above, the design defines a static channel that is based on the data provided by Kamstrup. In practice, this means that the SNR values between the nodes did not need to be computed. This, apart for being out of scope of the project, would be time consuming and impractical. What had been done instead, is to use the SNR values from Kamstrup's local list files in order to create an adjacency matrix. Every routing module has an local list file (for details see chapter \ref{sec:data_description}) that contains the SNR values from all the nodes it can ``hear''. Using these files, the matrix is populated and used as follows:

\begin{itemize}
\item For every SNR value compute the cost of the connection with the cost function (see section \ref{sec:cost_function}) and connect the two nodes.
\item A connection between two nodes is represented by a an entry in the adjacency matrix not equal to zero. The nodes can be seen as neighbours.
\item When a node is transmitting the channel assures that the frame will be propagated to all the neighbours, while also introducing delays.
\end{itemize}

In practice this means that every node that is in the range of the transmitting node will receive the frame. The range is not defined as a distance around a node, but by the SNR values contained in the local lists of the routing modules. 

Simple meters don't have the ability to perform SNR measurements consequently there are no local list files available for this specific kind of nodes. At first glance this seems like an obstacle, not difficult to overcome though since it is known that they can never act as gateways (intermediate nodes) on a routing path. The solution lies in using the SNR measurements from the routing modules. These have information about all the neighbours, including the meters.

%--Provides the functionality of sending the data to evreyone in range.
To keep it simple, the channel assures that every node in range will receive a copy of the original frame. On that part, it is abstracted from some important communication aspects such as packet loss due to bit errors. In practice, this can be interpreted as the channel providing the same link quality between the transmitter and any of the nodes in his range, regardless of the distance or the SNR magnitude. The measured SNR values are not disregarded though. They are taken into consideration during the setup of the model while creating  the tree with the routing paths for each node. There, the SNR is part of the cost function to decide the links that comprise the routes.

%Adds the delays by considering sending time and propagation time
One way of interpreting the objective that is set by the Fast Data Collection protocol is to utilize the channel as efficiently as possible. One of the approaches chosen towards this goal is to transmit as many frames as possible at the same time, without letting each of them be noise to the rest. The way to accomplish that is to implement and apply the collision avoidance techniques described in previous chapters. Their basic functionality is to check if there is a transmission occupying the channel within range. This means the transmissions cannot be instantaneous, they must have a duration that corresponds to the real time of channel busy time. Using this duration it can be assured that the medium  is allocated only to this one transmitter in the area.

The total time a transmission affects and consequently allocates the medium (for any of the collision avoidance nodes according to Fast Data Collection protocol) is a result of two different mechanisms that introduce two distinct and independent to each other times. The first, which has the greatest impact on the overall delay is the hardware's specifications. Namely, the data rate in such power-constrained devices is bound to be low in favour of the power consumption and the necessity of keeping the bit error rate to a minimum - which reduces the exchange of redundant frames for retransmission in case of corrupted packets. The second time that affects the disturbance of the medium is determined by the network planning (placement of the nodes) and the laws of physics that describe the propagation of the electromagnetic field. Telecommunications refer to it as propagation time or propagation delay when it is mentioned with regards to the time it takes for a bit to travel from the transmitter to a specific receiver.
\\Hence the total allocation time of the medium at every position of the area is a result of the summation of transmission and propagation time.

\figur{0.5}{totalDelay}{Total delay of a transmission over a medium}{fig:totalDelay}

The transmission time for a frame is the total amount of time that the network interface needs to send the entirety of the packet's bit stream on the medium. The speed that determines how fast this process is performed is the data rate which is a measure of the number of bits that can be created per second.\\
The propagation time is, as mentioned earlier, the time it takes for a bit to travel the distance between transmitter and receiver. The propagation speed depends on the physical characteristics of the medium and more specifically its dielectric constant. In vacuum the propagation speed is equal to the speed of light. It is roughly the same for the air which is the medium in this case.\\
The total time it takes for a whole frame to reach a node is the time it takes for the bits to be created (starting with creating the first bit) plus the time to travel to its destination (stopping at the last bit received). The time a frame is in the air can be calculated following the same procedure. The difference is now, that in such a case the propagation time is computed in order to the maximum propagation distance.
%When we need to calculate the time a frame exists in the medium we follow the same procedure. The difference is that in such case we use the maximum propagation distance to calculate the propagation time.

\figur{0.4}{delays}{Propagation delay and transmission duration contribute to the total delay of a packet transmission between two nodes}{fig:delays}

%The functionality of checking whether the channel is occupied is implemented as part of the channel since it the module that can hold information about all the transmissions happening throughout the network.
At this point it had been analysed the way to calculate the total duration of a transmission. Next step is to include this information in our model. This had been implemented by keeping an updated list of all the transmissions occurring at every moment throughout the network as part of the channel model. Whenever a node transmits a frame, it gets registered in the list so that the channel is aware of it.\\
A collision avoidance enabled node is using a functionality which is implemented as part of the channel model (since it holds all the information on every transmission) in order to check whether the medium is occupied or not. As a first step, a clean up of the list is performed by removing the transmissions that have expired, meaning those whose existence exceeded the maximum calculated duration. The second step is to check whether any of the neighbouring nodes is among the ones registered in the transmission list. It is then the node's responsibility to back off when a neighbour is occupying the medium and try again at a later time.

\figur{1}{csma}{Sequence of events when a collision avoidance node is trying to transmit a packet}{fig:csma}
% Since it is design, a paragraph could be added with time scheduling according to the area and performing the sequential data collection before of after broadcasting to the next levels